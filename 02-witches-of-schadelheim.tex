%!TEX root=./witches-of-schadelheim.tex
\chapter{The Witches of Sch{\"a}delheim}\label{ch:adventure}

This chapter presents the full adventure, \textit{The Witches of
Sch{\"a}delheim}. It's divided into five \textit{Acts}, each of which contains
several \textit{Scenes}. The separate acts will all progress in order, but
the scenes within an act may occur in a different order from what is written,
based on the heroes' actions. Each scene will give an indication of how the
adventure could progress.


\section{Act One: Down the River Reik}\label{act1}
The adventure begins with the heroes journeying downriver, \textit{en route} to
some destination. Unfortunately for them (but fortunately for the plot) their
journey will be interrupted, and they'll be forced to make a stop in the town
of Sch{\"a}delheim, where they'll meet some rather interesting people\ldots

\subsection{On Board the \textit{Bessie Blue}}\label{act1scene1}
The opening scene of the adventure has the heroes sat on a river barge sailing
down the Reik in the direction of Altdorf. Whether the heroes knew each other
before embarking, or whether they've met on board, is something for the players
to decide amongst themselves. The nature of where they're all headed is also
up for discussion---not that it really matters, because their journey will be
interrupted long before they get there. The following should give the players an 
introduction to their current setting.

\begin{callout}
You've just set sail down the Reik on a small sailbarge called the `Bessie
Blue', carrying a load of logs out to Marienburg. It's a cramped vessel, and the
captain was quite clear that he expected you to keep out of his way as much as
possible, so really your only options are huddling at the front of the barge
for fresh air, or huddling around the small coal burner below deck for warmth.
Autumn in the Reikland is quite chilly and damp, and you've been grateful for
the warmth thus far. The small vessel has a single mast with a triangular sail,
and most of the deckspace is taken up by its shipment of logs, lashed down and
covered with oiled canvas. There's a narrow gap either side that gives room to
get to the bow, but the railing is only knee-height.

The plan is to moor up every night at various riverside inns. The bargeswain
certainly has no intention of feeding you, and there's only just enough room for
him and his mate to sleep on board, so it's just as well you'll be stopping.
Nightfall is still several hours away, and the day is yours to use as you like.
\end{callout}

There are several different things the PCs might like to do here. If this is
their first adventure together, then this is an excellent time to encourage the
players to have their characters introduce themselves to each other, and do some
character building. This can just play out as they like.

\begin{gmnote}{So you all meet in a tavern\ldots{}}
If this isn't the players' first adventure together, this scene really shouldn't
take up much time. There isn't a great deal to be learned about the people on
the barge, and they aren't intended to play a major role later in the adventure;
they're simply the means of delivering the characters to the plot. Feel free to
skip straight ahead to \textbf{Inn For a Shock}, below---or, feel free to let
them invent plot for you!
\end{gmnote}

Some might want to get to know the barge's captain, or to try and `explore' the
barge as much as they can. There isn't much to see on the barge: it's 103ft long
with a single mast in the middle. The top deck is long and low, with mounds of
logs tied together on top and with a narrow path down either gunwale leading to
the bow. There's some small space in the bow where people can stand, but often
that's where the mate sits on lookout. The cabin in the rear goes down below
deck and is where the coal burner sits, and is also where the captain and his
mate sleep at night. There's a half-height deck below the top deck where stores
are kept, but there's really nothing of interest to the PCs down there: just
bits and pieces of shipping supplies like rope and tar.

The captain's name is Hans Kaufmann, and he doesn't particularly want to talk.
He took the characters' money, but more out of need than desire. He's in his
early 40s, and is a rotund man with a thick black beard and a limp in his right
leg. If characters can pass a \textbf{Difficult \mbox{(-10)} Charm} Test they can
get him to at least stop scowling when they're around, and even share a few
words about the weather. A further \textbf{Difficult (-10) Charm} Test will
get him to open up a little about himself. He's been a bargeman all his life,
travelling the Reik between Altdorf and Marienburg carrying whatever anyone pays
him to. His limp is from an injury caused by goblin river pirates a number of
years ago. Characters with an appropriate maritime background (e.g., Seamen,
Riverfolk, Priests of Mannan) gain +10 to the initial charm test to reflect
Kaufmann's grudging respect for other people of the water; he still doesn't
particularly want to open up, however.

Allow this scene to play out until the characters run out of things they're
interested in learning about. When they're done, move straight on to the
next section. Hans Kaufmann doesn't really play any role later in the adventure,
so if you're pressed for time or the players are bored this section needn't take
much time at all.

\subsection{Inn For a Shock}\label{act1scene2}
The second evening of the \textit{Bessie Blue's} journey sees her mooring at
a riverside inn. As with the previous night, the characters will be expected to
secure their own lodgings in the inn while the two crewmen sleep on board.
However, events during the night will keep them from getting a good night's
sleep. Read the following to the players.

\begin{callout}
Evening of the second day of travel sees you pulling up to a riverside inn
called the `Trout and Pike'. Calling it an ``inn'' is giving it a little too much
credit however; it's a small two-storey building with a thatched roof, visibly
subsiding slightly on one side. That said, there's a welcoming light in all the
windows, and as the Bessie Blue pulls up to the jetty, Kaufmann calls
out a greeting to a woman standing in the doorway that could almost be called
friendly. There are two other barges of a similar size to the `Bessie Blue'
moored on other jetties, with lanterns hanging.
\end{callout}

The innkeeper is known as Frau Mara, and she is actually one of Kaufmann's
oldest friends. She's in her late 40s, and has known the bargeswain for more
than two decades. The \textit{Trout and Pike} is really just her own home,
which she uses to make some income following her husband's death several years
ago. She earns enough to feed her guests well enough, and also to employ a
young man to run errands around the place. It has a small common room downstairs,
and a total of four guest rooms. Food and board will cost the characters 5/- per
person per night, though Mara can be haggled down to a minimum of 2/6 with a
successful \textbf{Challenging (0) Haggle} Test. Alternatively, she will offer
characters free lodgings if they're willing to run errands around the inn
(chopping wood, work in the kitchen, etc.).

The crew of the other barges are already inside. Conversation is good-natured,
but quiet; Frau Mara is well respected and her inn is the only riverside inn for
miles in either direction, so no one wants to cause too much trouble inside.
The characters can spend their evening as they like; meals for characters who
aren't working cost 6d.

\begin{gmnote}{Inn Patrons}
There's nothing particularly noteworthy about the patrons of the Trout and
Pike---they're bargemen similar in status to captain Kaufmann. If you want
some ideas to use as the players interact with them, here are some suggested
characters:
\begin{itemize}
    \item \textbf{Conrad van Stiel} is a Marienburger travelling upstream to Altdorf
            with a shipment of Bretonnian wine. He sits quietly with his two mates and
            retires early.
    \item \textbf{Stefan Gr{\"u}ber} is an old competitor of Master Kaufmann. They
            often compete for the same shipments, and Kaufmann seats himself on the
            opposite side of the inn to him.
\end{itemize}
\end{gmnote}

Eventually, the characters will head to bed in whatever rooms they've
decided to purchase. They probably won't set a watch, given that they're at
an inn, which is all to the good. If anyone does decide to stay up through
the night, then a \textbf{Very Difficult (-20) Perception} test at about
2am will let them notice shadowy shapes clambering over the decks of the
barges, caught in the lantern light. If no one is awake to see that, all the
characters are awoken when one of the bargemen shouts an alarm as they come to
grips with the goblin pirates attmpting to capture the cargo on the barges.
The goblins had crossed from the other side of the Reik in makeshift rafts,
with the intention of murdering the crew as they slept and stealing the goods.

The full raiding party is far more than the PCs could be expected to fight.
Assume that the players will rush towards the \textit{Bessie Blue} and play
out a fight there, dropping bits of narration that suggest they can hear
similar goings on on the other barges. Stat blocks for the goblins are
provided. The goblins should outnumber the PCs 2:1, but once half of them have
ben put down the rest will flee. The numbers are more there to communicate
the sense of danger; the tight confines of the barges make it difficult for
all the goblins to come and fight at once.

\statblock{Goblin Pirate}
    {4 & 25 & 35 & 30 & 30 & 20 & 35 & 30 & 30 & 20 & 20 & 11}
    {Animosity, Armour~+1, Afraid (Elves), Infected, Night Vision, Weapon~+7}
    {}{}{}{}{}{}
